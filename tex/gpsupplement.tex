\pragmaonce

% adapted from https://www.overleaf.com/learn/latex/Commands
\newcommand{\dissertationelse}[2]{%
% adapted from https://tex.stackexchange.com/a/33577
\ifdefined\DISSERTATION
#1
\else
#2
\fi
}


\section{Genetic Programming Experiments}
\label{sec:gpsupplement}


\subsection{SignalGP}

SignalGP (Signal-driven Genetic Programs) is a GP representation that enables signal-driven (\textit{i.e.}, event-driven) program execution.
In SignalGP, programs are segmented into modules (functions) that may be automatically triggered by exogenously- or endogenously-generated signals.
Each module in SignalGP associates a tag with a linear sequence of instructions.
SignalGP makes explicit the concept of signals (events), which comprise a tag and, optionally, signal-specific data.
Signals trigger the module with the closest matching tag (according to a given tag-matching scheme), using any signal-associated data as input to the triggered module.
SignalGP can handle many signals simultaneously, processing each in parallel.

The SignalGP instruction set, in addition to including traditional GP operations, allows programs to generate internal signals, broadcast external signals, and otherwise work in a tag-based context.
Instructions contain arguments, including an evolvable tag, that may modify the instruction's effect, often specifying memory locations or fixed values.
Instructions may refer to program modules using tag-based referencing; for example, an instruction may trigger the execution of a program module using the instruction's tag to specify which module to trigger.
SignalGP also supports genetic regulation with promoter and repressor instructions that, when executed, allow programs to adjust how well subsequent signals match with a target function (specified with tag-based referencing).

See \citepinappendix{lalejini2018evolving} for a more detailed description of the SignalGP representation. Additionally, see the GitHub repository for the SignalGP implementation used in this work \citepinappendix{lalejini_2020_3781295}.

\subsection{Changing-signal Task Description}

The changing-signal task requires programs to express a distinct response to each of $K$ environmental signal (each signal has a unique tag).
Programs express a response by executing one of $K$ response instructions.
Successful programs can `hardcode' each response with the appropriate environmental signal, ensuring that each environmental signal's tag best matches the function containing the correct response.
We expect the particular metric used to match tags to influence how well programs adapt to changing-signal task.

During evaluation, we afford programs 64 time steps to express the appropriate response after receiving a signal.
Once a program expresses a response or the allotted time expires, we reset the program's virtual hardware (resetting all executing threads and thread-local memory), and the environment produces the next signal.
Evaluation continues until the program correctly responds to each of the $K$ environmental signals or until the program expresses an incorrect response.
During each evaluation, programs experience environmental signals in a random order; thus, the correct \textit{order} of responses will vary and cannot be hardcoded.

% Experiment overview
For each metric, we evolved 200 replicate populations (each with a unique random number seed) of 500 asexually reproducing programs in an eight-signal environment ($K=8$) for 100 generations.
We identified the most performant tag mutation rate (from a range of possible mutation rates) for each metric to use in our experiment.
These data (and analyses) are available online in the GitHub repository that houses these experiments \citepinappendix{lalejini_2020_3781295}.
We used the following per-bit tag mutation rates for the changing-signal task: 0.01 for the Hamming and Streak metrics, 0.002 for the Hash metric, and 0.02 for the Integer and Bidirectional Integer metrics.
Aside from tag mutation rate, the overall configuration used for each metric was identical.
We limited tag variation in offspring to tag mutation operators (bit flips) by initializing populations with a common ancestor program in which all tags are identical and by disallowing mutations that would insert instructions with random tags.

The full configuration details for the changing-signal task (including a guide to running these experiments on your local machine) can be found in the associated GitHub repository \citepinappendix{lalejini_2020_3781295}.

\subsection{Directional-signal Task Description}

% Task overview
As in the changing-signal task, the directional-signal task requires that programs respond to a sequence of environmental cues; in the directional-signal task, however, the correct response depends on previously experienced signals.
In the directional signal task, there are two possible environmental signals --- a `forward-signal' and a `backward-signal' (each with a distinct tag) ---  and four possible responses.
If a program receives a forward-signal, it should express the next response, and if the program receives, a backward-signal, it should express the previous response.
For example, if response-three is currently required, then a subsequent forward-signal indicates that response-four is required next, while a backward-signal would instead indicate that response-two is required next.
Because the appropriate response to both the backward- and forward-signals change over time, successful programs must regulate which functions these signals trigger (rather than hardcode each response to a particular signal).

% Evaluation overview
We evaluate programs on all possible four-signal sequences of forward- and backward-signals (sixteen total).
For each program, we evaluate each sequence of signals independently, and a program's fitness is equal to its aggregate performance.
Otherwise, evaluation on a single sequence of signals mirrors that of the changing signal task.

% Experiment overview
We used an identical experimental design for the directional-signal task as in the changing-signal task.
However, we evolved programs for 5000 generations (instead of 100) and re-parameterized each metric's tag mutation rate (these data are available in the associated GitHub repository \citepinappendix{lalejini_2020_3781295}):
0.001 for the Hamming and Hash metrics, 0.002 for the Integer and Streak metrics, and 0.0001 for the Bidirectional Integer Metric.

The full configuration details for the directional-signal task (including a guide to running these experiments on your local machine) can be found in the associated GitHub repository \dissertationelse{\citep{lalejini_2020_3781295}}{\citepinappendix{lalejini_2020_3781295}}.

\subsection{Data analysis and Implementation}

The source code for our GP experiments can be found in the following GitHub repository: \dissertationelse{\citep{lalejini_2020_3781295}}{\citepinappendix{lalejini_2020_3781295}}. This repository additionally includes all data analysis and visualization scripts, experiment configuration details, and a guide for running our experiments locally.
