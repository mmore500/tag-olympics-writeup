\section{Methods}

\subsection{Distance Metrics}

\begin{itemize}
\item Positive integers with wraparound \citep{spector2011tag}
\item Streak method \citep{downing2015intelligence}
\item Hamming distance \citep{lalejini2019else}
\item NK-fitness landscape \citep{kauffman1987towards} (note: this isn't directly symmetric tag-tag matching)
\item full-on protein docking simulations (or at least cite the BEACON person who's working on this
\item Anselmo's proprietary tags (maybe if we promise to keep it proprietary?)
\item uneven weighting of 0/1 probabilities at different locations on the bitstring (MAM)
\end{itemize}

a list of bytes, must have at least X match within each byte (personal communication with Ofria)
\begin{itemize}
\item min
\item max
\item multiplicative
\item additive
\end{itemize}

meta... multi-receptor tags (e.g., multiple receptors you can match... variable number?) (my idea)
\begin{itemize}
\item multiplicative
\item min
\item additive distance among sub-tags
\item max
\end{itemize}

Multi-level/hierarchical tags: i.e., you need to reference down a hierarchy. For
example, access a member of a member of a member of a particular struct.
\begin{itemize}
  \item If you know how many levels, you can use different `chunks' of the tag for
        each level.
  \item Use the same tag at each level (let evolution work out the `chunking')
  \item Source multiple tags from the referer (e.g., need to go down three levels?
    be sure to get three tags, one for each level)
\end{itemize}

\subsection{Selection Techniques}

best with threshold \citep{lalejini2019else}

roulette
\begin{itemize}
\item skew
\item similarity cut-off
\end{itemize}

\subsection{Evolutionary Characteristics}

find citations for this subsection in my undergraduate thesis

\subsubsection{Duplication and Differentiation}

Question: How fast/complicated can duplication/divergence events occur?

\subsubsection{Canalization}

Question:
\begin{itemize}
\item several sets of ``related" signal/response groups... triggering the wrong signal/response pair within the group is okay... but triggering the signal/response pairs that correspond to different groups is not okay
\item do we end up with mutations tending to cause changes that stay within the group? how much mutation until we start to see signal/response pairs from different groups triggered
\item (side question) how strong of an effect can plasticity have on promoting canalization (e.g., select for plastic rearrangement of connections within group)
\end{itemize}

\subsubsection{Plasticity (via regulation)}

Question: how fast/how complicated can we evolve plastic responses (e.g., in environment A certain signal/response pairs; in environment B certain signal/response pairs) using regulation

maybe also differentiation?
(e.g., go stably into state A or state B based on an initial environmental cue)

\subsubsection{Bandwidth}

Question: what is the relationship between the number of signals/responses and the evolutionary difficulty of
\begin{itemize}
\item evolving n connection pairs de-novo
\item with n connection pairs evolved, evolving a n+1 connection pair
\end{itemize}

\subsubsection{Hidden Genetic Variation}

Question: evolve one signal/response pair (might need multiple signal/response pairs), how much tag diversity exists in population at end of run?

\subsubsection{Degeneracy}

Question: how many independent but redundant cue/response tag pairs (that link the same cue and same response) arise spontaneously?

\subsubsection{Robustness}

\begin{itemize}
\item plot match score versus random mutational walk like in Downing
\item fat-tailed or thin-tailed? (e.g., gradual or walking off a cliff)
\end{itemize}

distribution of effect on match score for different mutations
\begin{itemize}
\item are some mutations silent?
\item do some mutations cause extreme/catastrophic effects?
\end{itemize}

\subsection{System-level Metrics}

systems:
\begin{itemize}
\item environment matching with SignalGP
\item DISHTINY with SignalGP
\item lawnmower \citep{spector2011tag} / dirt-sensing, obstacle-avoiding robot \citep{spector2011tag} / even-4-parity \citep{spector2012tag}
\item TODO more / better please
\end{itemize}

metrics:
\begin{itemize}
\item evolvability signatures \citep{tarapore2015evolvability}
\item solution quality
\item phenotypic lock-in
\item the graph structure of gene regulatory networks that tend to evolve?
\end{itemize}

\subsection{Computational Efficiency}

\begin{itemize}
\item time to calculate distance between two tags
\item the computational complexity of selection techniques
\item possible optimizations that make matching better than linear?
(not with regulation, I suspect)
\end{itemize}

\subsection{Applications}

\begin{itemize}
  \item GP Modules (SignalGP, Lee's PushGP stuff)
  \item Memory access in GP: use tags to label and refer to locations in memory.
        Tags should allow us to evolve the \textit{size} of our memory buffer. Perhaps
        this is useful?
  \item Neural networks: could use tags to specify (and evolve) connections between
        nodes and to specify node functions (e.g., 0000 = sin, 1111 = threshold, etc)
\end{itemize}

\subsubsection{SignalGP Tag Extensions}

\begin{itemize}
  \item Module regulation
  \item Allow modules to be renamed by `renaming' instructions (another form of
        module regulation). This is sort of similar to Lee's PushGP implementation
        that allows Push programs to label and subsequently relabel tagged things.
\end{itemize}

\subsection{Implementation}

We implemented our experimental system using the Empirical library for scientific software development in C++, available at \url{https://github.com/devosoft/Empirical}.
The code used to perform and analyze our experiments, our figures, data from our experiments, and a live in-browser demo of our system is available via the Open Science Framework at \url{https://osf.io/TODO/}.
