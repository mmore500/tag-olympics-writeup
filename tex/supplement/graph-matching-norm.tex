\section{Graph Matching with Normally-distributed Mutation Operator}
\label{sec:graph-matching-norm}

In order to contextualize our use of bitwise mutation with integer metrics, we replicated the 32-node target graph matching experiment with randomly-initialized genomes reported in Section \ref{sec:graph-matching} using a normally-distributed mutation operator.

Under the normally-distributed mutational operator, each tag had an integer amount drawn from a normal distribution added to its integer representation every generation.
Specifically, the amount added was drawn from $\mathcal{N}(0, m \times c)$, where $m$ was a variable mutation rate parameter and $c$ was a fixed scale parameter $2^{32} - 1$.
(Recall that $2^{32} - 1$ is the largest possible
Overflow values after addition were wrapped around.

A swath of 15 standard deviations $m$ between 0.000976562 and 0.125 were surveyed.
All metrics had a best-performing mutation rate within the range of surveyed rates for each problem configuration tested.
Supplementary Figure \ref{fig:evolve_zeroinit_mutsweep} summarizes performance across surveyed mutation rates.
We used the best-performing mutation rate for each metric on each problem configuration for further analysis.
Supplementary Table \ref{tab:evo_graph_mut_norm} provides the best-performing mutation rates used.

Supplementary Figure \ref{fig:evolve_bests_norm} shows adaptation over generations under this mutational operator for regular/irregular target graphs with mean degree 1/2.
The normally-distributed mutation operator performs comparably to the bitwise mutation operator or worse in all instances.

It may be possible to achieve better performance with a mutation operator that combines a per-tag mutation probability with a normally distributed mutational effect.
Further work will be required to explore such possibilities.
