\section{Hash Metric} \label{sec:hashmetric}

This metric is original to the our paper and meant to serve as a control.
The metric yields an arbitrary, but determinsitic, match distance between two tags.
Match distance is uniformly distributed between 0 and 1 for randomly-sampled tag pairs.

To compute the hash metric, we begin by taking a SHA-1 digest of a concatenation of the tags' bit representations.
We refer readers to the literature for details of the SHA-1 algorithm \citep{eastlake2001us}.
This process yields a 20 byte array of \texttt{unsigned char}.
Then, we use the \texttt{std::hash} utility to convert a \texttt{std::string} initialized with this data to a \texttt{std::size\_t}.
Finally, we cast and divide by \texttt{std::numeric\_limits<size\_t>::max()} to yield match distance as a \texttt{double}.

This metric is not commutative.
As noted in Section \ref{sec:introduction}, however, tag-matching systems inherently distinguish queries and operands.
So an ordering within each pair of tags processed in a tag-matching system will be well-defined.
We use the convention of ordering the operand tag after the query tag when concatenating the tags' bit representations.
