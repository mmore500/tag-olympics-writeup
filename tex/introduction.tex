\section{Introduction}

TODO \citep{taylor2016open}

\subsection{What is a tag?}

At a broad level, tags are labels that can be used to refer to labelees (the
tagged thing).

\subsubsection{Examples of tags}

\begin{itemize}
  \item Human culture (probably don't want to draw too many examples from this category)
    \begin{itemize}
      \item We name our offspring, which is useful for later referring to them.
            Names are often repeated across individuals (e.g., we'll be seeing an
            uptick in `Arya's post-GoT). Context + multi-level naming (first, middle,
            last in 20th century American culture) is important for resolving collisions.
    \end{itemize}
  \item Software development
    \begin{itemize}
      \item In software development, we label and refer to lots of things! In this
            domain, tags are precise: after labeling (tagging) something, subsequent
            referrals to those labeled things must be \textit{exact}. Inexactness
            in a reference typically results in syntactic errors.
      \item Common uses: labeling and specifying locations in memory, custom data
            structures (e.g., structs, classes, etc), functions, libraries/modules,
            language constants/built-ins, program entry points, locations in an
            instruction sequence (e.g., function names, looping, conditionals etc).
    \end{itemize}
  \item Chemistry?
  \item Biology?
    \begin{itemize}
      \item cell signaling
      \item protein folding
    \end{itemize}
\end{itemize}

\subsubsection{What are the benefits of tags/tag-based referencing?}

\begin{itemize}
  \item Hypothesis: Inexactness allowed by tag-based referencing makes these references
        more robust to minor genetic perturbations, smoothing the genotype-phenotype
        mapping relative to more traditional memory-indexing techniques (pulled from
        2019 Tag-access memory abstract).
  \item We don't need to know/lock-in the architecture of what our tags are referencing.
        If a referent (e.g., module) is deleted, it doesn't invalidate any of the
        in-program references (e.g., module calls). The same is true for creating
        a new referent. For example, using tags to reference program modules allows
        you to mutate the number of modules in the program without (necessarily)
        breaking existing references.
  \item Hypothesis: Tag-based referencing should help to enable the duplication/
        deletion of referents (e.g., modules), which should improve capacity for
        complexity to evolve (i.e., duplication is often cited as important in the
        evolution of complex features).
\end{itemize}
