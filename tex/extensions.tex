\section{Extensions \& Future Work}


GOOD IDEAS
* uniformity/asymmetry
  * promiscuity (node connectivity in small world networks distributed by power law)
  * modularity / hierarchy (e.g., further specification Holland)


BLAH BLAH BLAH
In the process of preparing these experiments we generated a number of ideas that were outside the scope of the current work.
TODO

\begin{itemize}
\item NK-fitness landscape \citep{kauffman1987towards} (note: this isn't directly symmetric tag-tag matching)
\item full-on protein docking simulations (or at least cite the BEACON person who's working on this)
\item Anselmo's proprietary tags (maybe if we promise to keep it proprietary?)
\item uneven weighting of 0/1 probabilities at different locations on the bitstring (MAM)
\end{itemize}

a list of bytes, must have at least X match within each byte (personal communication with Ofria)
\begin{itemize}
\item min
\item max
\item multiplicative
\item additive
\end{itemize}

meta... multi-receptor tags (e.g., multiple receptors you can match... variable number?) (my idea)
\begin{itemize}
\item multiplicative
\item min
\item additive distance among sub-tags
\item max
\end{itemize}

Multi-level/hierarchical tags: i.e., you need to reference down a hierarchy. For
example, access a member of a member of a member of a particular struct.
\begin{itemize}
  \item If you know how many levels, you can use different `chunks' of the tag for
        each level.
  \item Use the same tag at each level (let evolution work out the `chunking')
  \item Source multiple tags from the referer (e.g., need to go down three levels?
    be sure to get three tags, one for each level)
\end{itemize}

\subsection{Evolutionary Characteristics}

find citations for this subsection in my undergraduate thesis

\begin{itemize}
  \item Question: do tag matches evolve to become more specific? i.e., become stronger over time?
        If so, do matches lock-in early, leading to strong historical contingencies?
  \item Question: How stable are tag-based references over time? Are stable references
        associated with successful lineages? Does stability change over time (e.g.,
        low stability early, increasingly higher stability over time? W/major stability
        shake-ups when new architectures arise)
  \item Do tags enable more complex genetic architectures over traditional reference
        techniques?
  \item Quantify how often perturbations (mutations) change references. Compare
        successful lineages to general population averages (are successful lineages
        avoiding or embracing perturbation)
  \item Track the frequency of and effect of (deleterious, neutral, beneficial)
        perturbations that change references. Again, are successful lineages different
        from population averages?
\end{itemize}

\subsubsection{Duplication and Differentiation}

\begin{itemize}
  \item Question: How fast and complicated can duplication/divergence events occur?
        Quantify how often these events occur in general vs how often they are along
        successful lineages.
\end{itemize}

\subsubsection{Canalization}

Question:
\begin{itemize}
\item several sets of ``related" signal/response groups\ldots triggering the wrong signal/response pair within the group is okay\ldots but triggering the signal/response pairs that correspond to different groups is not okay
\item do we end up with mutations tending to cause changes that stay within the group? how much mutation until we start to see signal/response pairs from different groups triggered
\item (side question) how strong of an effect can plasticity have on promoting canalization (e.g., select for plastic rearrangement of connections within group)
\end{itemize}

\subsubsection{Plasticity (via regulation)}

Question: how fast/how complicated can we evolve plastic responses (e.g., in environment A certain signal/response pairs; in environment B certain signal/response pairs) using regulation

maybe also differentiation?
(e.g., go stably into state A or state B based on an initial environmental cue)

\subsubsection{Bandwidth}

Question: what is the relationship between the number of signals/responses and the evolutionary difficulty of
\begin{itemize}
\item evolving n connection pairs de-novo
\item with n connection pairs evolved, evolving a n+1 connection pair
\end{itemize}

\subsubsection{Hidden Genetic Variation}

Question: evolve one signal/response pair (might need multiple signal/response pairs), how much tag diversity exists in population at end of run?

\subsubsection{Degeneracy}

Question: how many independent but redundant cue/response tag pairs (that link the same cue and same response) arise spontaneously?

\subsubsection{Robustness}

\begin{itemize}
\item plot match score versus random mutational walk like in Downing
\item fat-tailed or thin-tailed? (e.g., gradual or walking off a cliff)
\end{itemize}

distribution of effect on match score for different mutations
\begin{itemize}
\item are some mutations silent?
\item do some mutations cause extreme/catastrophic effects?
\item e.g., how often do perturbations disrupt existing references?
\end{itemize}

\subsection{System-level Metrics}

systems:
\begin{itemize}
\item environment matching with SignalGP
\item DISHTINY with SignalGP
\item lawnmower \citep{spector2011tag} / dirt-sensing, obstacle-avoiding robot \citep{spector2011tag} / even-4-parity \citep{spector2012tag}
\item TODO more / better please
\end{itemize}

metrics:
\begin{itemize}
\item evolvability signatures \citep{tarapore2015evolvability}
\item solution quality
\item phenotypic lock-in
\item the graph structure of gene regulatory networks that tend to evolve?
\end{itemize}

\subsection{Computational Efficiency}

\begin{itemize}
\item time to calculate distance between two tags
\item the computational complexity of selection techniques
\item possible optimizations that make matching better than linear?
(not with regulation, I suspect)
\end{itemize}

\subsubsection{SignalGP Tag Extensions}

\begin{itemize}
  \item Module regulation
  \item Allow modules to be renamed by `renaming' instructions (another form of
        module regulation). This is sort of similar to Lee's PushGP implementation
        that allows Push programs to label and subsequently relabel tagged things.
\end{itemize}
