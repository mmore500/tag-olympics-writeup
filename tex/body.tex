\section{Introduction} \label{sec:introduction}

Genetic programming often requires operations to undergo evolutionary adjustment of which operands they act upon.
This process can tweak the semantics of existing evolved code, critical in particular for duplication and divergence processes commonly highlighted in discussions of evolvability \citep{altenberg1994evolution}.
This process also allows for incorporation and removal of operands.
Additionally, dynamic reorganization of code modules can facilitate hierarchical problem-solving \citep{Kinnear:Koza:1994:adf}.

Tag-based referencing, also known as ``pattern matching'' or ``inexact referencing,'' offers a practical solution for choosing computational operands.
This approach attaches a tag to each computational operand and query.
Operands are then selected for each query according to compatibility between respective tags.
For example, consider a system where matching is performed according to absolute difference between integer tags.
If two operands tagged 8 and 2 are available, a query tagged 7 would retrieve the operand tagged 8.
Conversely, a query tagged 2 would retrieve the operand tagged 2.

Inexact referencing techniques are widely used in agent-based modeling \citep{riolo2001evolution}, neuroevolution \citep{reisinger2007acquiring}, artificial gene regulatory networks \citep{banzhaf2003artificial}, and genetic programming \citep{spector2011tag, lalejini2018evolving}.
While some attempts have been made to distinguish tag-matching criteria through narrative explanations of evolutionary properties and biological analogies \citep{downing2015intelligence,scherer2004activation}, our work is the first to provide systematic, quantitative, and empirical insight into the consequences of commonly used tag-matching criteria.

\section{Methods}

Rigorous comparison of disparate tag-matching schemes required careful standardization of tag representation, mutation, and match scoring.
We used 32-bit bitstrings as tags for all experiments.
Mutations toggled individual bits stochastically at a uniform per-bit rate.
Match distances were normalized to ensure a uniform distribution of scores between randomly sampled tags.
This ensured consistent and intuitive interpretation across all tag-matching metrics.

\begin{table*}[!htbp]
\begin{tabularx}{\textwidth}{l|X}
\textbf{Metric}       & \textbf{Description}                                                                                                                                        \\ \hline
Hash                  & SHA1 cryptographic hash of  concatenation of \texttt{tag\_0} and \texttt{tag\_1} \citep{eastlake2001us}                         \\ \hline
Hamming               & fraction of positions within \texttt{tag\_0} and \texttt{tag\_1} with mismatching bits                                                                         \\ \hline
Integer               & value added to the unsigned integer representation of \texttt{tag\_0} to reach representation of \texttt{tag\_1}, wrapping around if necessary \\ \hline
Bidirectional Integer & lesser of integer metric distances \texttt{d(tag\_0, tag\_1)} and \texttt{d(tag\_1, tag\_0)}                                                                \\ \hline
Streak                & ratio of lengths of contiguously matching and mismatching substrings \\ \hline
\end{tabularx}

% \begin{tabularx}{\textwidth}{l|X|X}
% \hline
% \textbf{Metric}       & \textbf{Commutative?} & \textbf{Multidimensional?} \\ \hline
% Hash                  & no                    & yes                        \\ \hline
% Hamming               & yes                   & yes                        \\ \hline
% Streak                & yes                   & yes                        \\ \hline
% Integer               & no                    & no                         \\ \hline
% Bidirectional Integer & yes                   & no
% \end{tabularx}

\caption{
Surveyed tag-matching metrics.
}
\label{tab:metrics}
\vspace{-6ex}
\end{table*}


We compared five tag-matching metrics: Hamming, hash, integer, bidirectional integer, and streak (Table \ref{tab:metrics}).
We included the Hamming and bidirectional integer metrics because of their ubiquity in genetic programming.
We included the integer metric due to its use in prior tag-matching studies \citep{spector2011tag,spector2012tag}.
\cite{downing2015intelligence} proposed the streak metric to model large-effect mutations observed in biology but had not yet been formally studied.
We introduce the hash metric as a control that has no geometric structure, in contrast to all other metrics.

Implementations of all tag-matching algorithms were incorporated into the open-source Empirical C++ library \citep{charles_ofria_2019_2575607} as interchangeable components of the MatchBin tool suite.

\section{Summary of Results}

% We explored how these tag-matching schemes differ with respect to
% (1) \textit{geometric structure} that biases or limits the patterns of connectivity that form among queries and operands, (2) \textit{variational properties} that influence changes to connectivity observed under mutation, and (3) \textit{evolutionary outcomes} such as the rate of adaptive evolution and the quality of evolved solutions.

\subsection{Geometric and Variational Analyses}

We first investigated the impact of different tag-matching metrics on mutational neighborhoods and geometric constraints of connectivity patterns.
Single-step and multi-step mutational analyses allowed us to characterize the local and broader mutational landscapes induced by each metric.
The integer metrics displayed the most restrictive geometric structure, followed by the Hamming and streak metrics.
Such restrictive structure limits possible connectivity patterns between tagged components, for instance precluding a pair of query tags that both closely match one operand tag from strongly disagreeing in match affinity to other operand tags.

Single-step mutational analyses showed large-effect one-step mutations under the hash, integer, and streak metrics, but not under the Hamming metric.
For multi-step mutational analyses, match affinity decayed most rapidly along mutational walks under the integer metrics and the control hash metric.
Match affinity decayed slowest along mutational walks under the Hamming metric, while the streak metric degraded second-slowest.

Interestingly, our results contradict existing hypotheses of the streak metric's mutational properties, which propose greater robustness compared to the Hamming metric \citep{downing2015intelligence}.
This discrepancy arises from our normalization to ensure a uniform distribution of raw match scores.
We believe our result to be more representative, as normalized match distance corresponds to the probability that arbitrary tags would match more strongly by chance.
This directly relates to how effectively an operand tag competes to be the ``best'' match for a query.

\subsection{Evolutionary Experiments}

Next, we investigated the performance of various tag-matching metrics in digital evolution, starting with a toy problem and progressing to the more complex, applied domain of the SignalGP genetic programming representation.
In the toy problem, we define a target connection topology between tagged queries and operands.
We instantiate a population of tag sets and select for those with tag-match pairs that correspond to target topology components.
Across experiments, we systematically varied the target topological density, comparing scenarios with low constraint (i.e., each query selected to match with a single operand) and high constraint (i.e., each query selected to match with multiple operands).

In the SignalGP problem domain, tag-based referencing facilitates activation of modules (functions) in response to exogenously- and endogenously-generated signals.
Tags specify the relationship between signals and signal-handlers (program modules), triggering the module with the closest matching tag to run its linear sequence of instructions.
The instruction set includes traditional GP operations, instructions to generate arbitrarily-tagged internal and external signals, as well as module promoter and repressor instructions to facilitate during-lifetime plasticity as described in \cite{lalejini2021tag}.

We perform experiments with two diagnostic SignalGP problems: the changing-signal task which selects for sparse tag interactions (low constraint) and the directional-signal Task which selects for denser tag interactions (high constraint).

The Hamming and streak metrics performed best in the SignalGP experiments.
In the high-constraint directional-signal task, the streak metric outperformed the Hamming metric.
The hash metric showed the next-best performance, yielding more solutions than the integer metrics, which both performed comparably poorly.

Interestingly, the hash metric performed best in low-constraint target-matching experiments, likely due to its strong variation-generation capacity.
However, the hash metric was outperformed in low-constraint SignalGP experiments by the streak and Hamming metrics.
Similarly, the integer metrics were outperformed by the streak and Hamming metrics in low-constraint SignalGP experiments, despite their good performance in low-constraint target-matching experiments.

It remains unclear what aspect of the genetic programming experiments selectively stymy the hash and integer metrics.
Possibilities include better streak and Hash support for duplication and differentiation processes along SignalGP lineages or differences in fitness landscape ruggedness.
Further experiments are needed to understand the evolutionary dynamics of non-trivial tag-matching systems and the interplay of those dynamics with tag-matching metrics.

\section{Conclusion}

In a first step toward systematic theory for evolutionary applications of tag-matching, we have cataloged mechanistic differences between tag-matching criteria and demonstrated significant impact of these differences on adaptive evolution.
This work provides practical guidance on tag-matching metrics and mutation rates that can be applied to existing genetic programming systems.
Important open questions remain, such as the relationships between tag-matching criteria and specificity, modularity, robustness, and the process of duplication and divergence.
Evolvability or information-theoretical analyses may prove fruitful in exploring these questions.
A framework for the systematic design of new tag-matching metrics and mutation operators with desirable evolutionary properties should also be pursued.

Tag-like mechanisms mediate interaction and function across various biological scales.
Investigating the mechanistic and evolutionary properties of tagging systems promises more nuanced understanding and algorithmic emulation of natural systems.
