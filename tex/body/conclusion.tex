\section{Conclusion}

Better understanding the mechanistic properties and functional implications of tag-matching criteria will enable more effective incorporation of tag matching in evolutionary systems.
Within genetic programming, tuning tag-matching criteria could facilitate faster evolution of higher-fitness solutions.
Here, tag matching approaches have been highlighted, particularly, for their potential to enable dynamic, modular reconfiguration of evolved programs at runtime \citep{spector2011tag,lalejini2021tag}.
Likewise, within artificial life, tuning tag-matching criteria could enable more novelty and complexity within evolving systems.
There has been interest, in particular, in the potential for tag-based referencing to facilitate inter-species interactions in digital ecologies \citep{dolson2021review}.

Our analyses suggests a key role of network constraint in the interaction between a tag-matching scheme and problem domain.
Applications where queries much match tightly with multiple operands require high-dimensional tag-matching criteria.

The surprisingly strong performance of the hash metric on low constraint toy problems highlights how tag-matching criteria can facilitate generation of phenotypic variation.

Important open questions remain with respect tag-matching criteria.
In particular, the relationships between tag-matching criteria and specificity, modularity, robustness, and the process of duplication and divergence should be explored.
Evolvability or information-theoretical analyses may prove fruitful in this regard \citep{tarapore2015evolvability}.
How to systematically design new tag-matching metrics with desirable evolutionary properties also remains an open problem.
We also need algorithms capable of performing fast look ups under high-dimensional or irregular tag-matching metrics, ideally achieving sublinear time complexity on large sets of referents.

Tag-like mechanisms play a central role mediating interaction and function across the spectrum of biological scale \citep{holland2012signals}.
By shining light on previously-unexplored mechanistic and evolutionary properties of tagging systems, we hope that insight into artificial tag models will translate into a more nuanced appreciation of natural systems.
