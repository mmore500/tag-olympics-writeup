\subsection{Streak Metric} \label{sec:streak}

The streak metric computes match distance between bitstring tags $t$ and $u$ as a ratio of lengths of contiguously matching and mismatching substrings within those tags.

Formally, we can compute the greatest contiguously-matching length of $n$-long bitstrings $t$ and $u$ as,
\begin{align*}
m(t, u) = \max\Big(\{i - j \forall i, j \in 0..n-1 \mid \forall q \in i..j, t_q = u_q \}\Big).
\end{align*}

Likewise, the greatest contiguously-mismatching length can be computed as,
\begin{align*}
n(t, u) = \max\Big(\{i - j \forall i, j \in 0..n-1 \mid \forall q \in i..j, t_q \neq u_q \}\Big).
\end{align*}

As proposed in \cite{downing2015intelligence}, the streak metric computes distance between $n$-bit bitstring tags $t$ and $u$ as,
\begin{align*}
d'(t, u)
= \frac{p'(n(t,u))}{p'(m(t,u)) + p'(n(t,u))}.
\end{align*}
where $p(k)$ approximates the probability of a contiguous $k$-bit match between two bitstrings.
\cite{downing2015intelligence} derives
\begin{align*}
p'(k)
= \frac{n - k + 1}{2^k}.
\end{align*}

However, this formula is subtly flawed.
For instance, the probability of a $0$-bit match according to this formula would be computed as $p'(0) = \frac{n - 0 + 1}{2^0} = n + 1$.
This is clearly impossible --- it would imply $p'(0) > 1 \forall n > 0$.

Although correct probabilities can be calculated via dynamic programming, $p'$ provides a useful approximation.
For computational efficiency and consistency with the existing literature, we use the math proposed in \citep{downing2015intelligence} but clamp edge cases between 0.0 and 1.0.
This yields the corrected streak metric $d$ used in this work,
\begin{align*}
d(t, u) = \max\Big( \min( d'(t, u), 1), 0 \Big).
\end{align*}

Downing motivates the streak metric by analogy to the biochemistry of enzyme-ligand binding.
In motivating the metric, Downing reports mutational walk experiments that show it to exhibit greater robustness compared to integer and hamming metrics.
However, it is not demonstrated in an evolving system.
To our knowledge, no further work on this metric has been published.
(Although, through personal communication, we learned of some unpublished work applying the metric in a neuroevolution system.)
