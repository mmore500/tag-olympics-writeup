\subsection{Hash Metric} \label{sec:hash}

To our knowledge, the hash metric is original to this work.
The metric produces an arbitrary, but deterministic, match distance between any two tags.
In other words, the tag matching space is completely unstructured.
We include it primarily to serve as a control.

The hash metric calculates match distance via a SHA1 cryptographic hash of tags $t$ and $u$ \citep{eastlake2001us}.
First, we concatenate $t$ and $u$ into a double-width bitstring $v$ such that
\begin{align*}
v = \langle t_0, t_1, t_2, \dots, t_{n-2}, t_{n-1}, u_0, u_1, u_2, \dots, u_{n-2}, u_{n-1} \rangle
\end{align*}

Then, we use the OpenSSL library to generate a \texttt{std::string} digest of $v$.
We then apply \texttt{std::hash} to map this digest to a \texttt{std::size\_t}, $v'$.
Finally, we perform a floating point division to compute the matching distance as $d(t, u) = v' / \hat{V}$ where $\hat{V}$  denotes the maximum representable \texttt{std::size\_t} value.

Note that this metric is not commutative.
As noted above, however, tag-matching systems inherently distinguish queries and operands.
So, an ordering within each pair of tags processed in a tag-matching system will be well-defined.
We use the convention of ordering the operand tag after the query tag when concatenating the tags' bit representations.

Take as an example the two eight bit tags $t = \langle 1, 0, 0, 1, 0, 0, 1, 1 \rangle$ and $u = \langle 0, 0, 1, 0, 0, 0, 1, 1 \rangle$.
The intricacy of the SHA1 algorithm precludes working through an example in full detail but these two tags will have an arbitrary match distance --- suppose it as 0.24.
Changing any bit in either of the tags $t$ or $u$ will completely scramble match distance.
For example, with the first bit of $t$ flipped match distance might instead be computed as 0.89.
Flipping the second bit instead could yield a totally different result.
