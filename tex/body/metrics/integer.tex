\subsection{Integer Metric} \label{sec:integer}

The integer metric computes match distance between tags $t$ and $u$ by counting upwards from $t$ until $u$ is reached.
If necessary, the counting process wraps around at $2^n$.

To accomplish this, the integer metric must interpret bitstring tags $t$ and $u$ as unsigned integers.
We use a standard representation,
\begin{align*}
f(t)
= \sum_{i=0}^{n-1} t_i \times 2^i.
\end{align*}

Formally, the integer metric computes distance between $n$-bit bitstring tags as,
\begin{align*}
d(t, u) = \frac{\Big(f(u) - f(t)\Big) \mod 2^n}{2^n}.
\end{align*}

Inclusion of this metric is motivated by \cite{spector2011tag}, who used positive integers between 0 and 100 to name referents.
Queries matched to the referent that had the next-larger value, wrapping around from 100 back to 0.

Like the hash metric, this metric is not commutative.
We adopt the convention of using the query tag as $t$ and the operand tag as $u$.

As an example, consider eight bit tags $t = \langle 1, 0, 0, 1, 0, 0, 1, 1 \rangle$ and $u = \langle 0, 0, 1, 0, 0, 0, 1, 1 \rangle$.
The tag $t$ encodes the integer value 147.
The tag $u$ encodes the integer value 35.
Computing the difference between these integers mod 256 yields 144.
This is equivalent to starting at 147 then counting up by 109 to reach 256, wrapping around to 0, then counting up a further 35.
Dividing by 256 to normalize gives the match distance 0.5625.
