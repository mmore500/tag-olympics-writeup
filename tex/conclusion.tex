\section{Conclusion}

Better understanding the mechanistic properties and functional implications of tag-matching criteria will help researchers more effectively incorporate tag matching in evolutionary systems and better understand the biases imposed by those criteria.
Within genetic programming, bespoke tag-matching criteria might increase the rate of adaptive evolution and evolving better-quality solutions.
Likewise, within artificial life bespoke tag-matching criteria might improve generation of novelty and complexity.

Our analyses suggests that network constraint is key to the interaction between a tag-matching scheme and problem domain.
Applications where queries much match tightly with multiple operands require high-dimensional tag-matching criteria.

The surprisingly strong performance of the hash metric on low constraint toy problems   underscores the role of tag-matching criteria in facilitating generation of phenotypic variation.

Important open questions remain with respect tag-matching criteria.
In particular, the relationships between tag-matching criteria and specificity, modularity, robustness, and the process of duplication and divergence should be explored.
Evolvability or information-theoretical analyses may prove fruitful in this regard \citep{tarapore2015evolvability}.
How to systematically design new tag-matching metrics with desirable evolutionary properties also remains an open problem.

Tag-like mechanisms play a central role mediating interaction and function across the spectrum of biological scale \citep{holland2012signals}.
By shining light on previously-unexplored mechanistic and evolutionary properties of tagging systems, we hope that insight into artificial tag models will translate into a more nuanced appreciation of natural systems.
