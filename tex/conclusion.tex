\section{Conclusion}

we have characterized different tag schemes and demonstrated how they can have evolutionarily-observable consequences.

build stronger theory (evolvability and information-theory analysis)
\begin{enumerate}
\item  apply to understanding properties of biological e.g., proteins
\item  build a toolbox of tag matching schemes informed by theory
\item  modularity, duplication and differentiation, other things we didn't look at etc. etc.
\end{enumerate}

incorporate modular tag matching into artificial life models: open-source C++ tag-matching data structure implementation in Empirical with interchangeable metrics and selection algorithms

Holland notes tags as modular signal boundary systems \citep{holland2012signals}
This has been identified as a larger context of complex systems with parallels drawn to biology.
Insight into simple, abstract tag models might translate into a more nuanced appreciation of the principles underlying enzymatic protein-protein matchining biological systems.

On tag-matching constraint:
In a dynamically matched query-to-single-operand system, however, this can be relevant to the resulting connectivity under runtime silencing or upregulation of modules.
However, even in a static system, this can be relevant to resulting connectivity under deletion of an operand.

