\section{Conclusion}

Better understanding the mechanistic properties and functional implications of tag-matching criteria will help artificial life practitioners more effectively incorporate tag matching in model systems and better understand the biases imposed by those criteria.
In particular, our analyses suggests that network constraint (i.e., the degree of connectivity networks constructed by tag matching) is key to the interaction between a tag-matching scheme and problem domain.
% Better understanding tag-matching criteria willOutside of artificial life, particularly in genetic programming, where increasing the rate of adaptive evolution and evolving better-quality solutions is a key concern.

However, open questions remain with respect tag-matching criteria.
In particular, the relationships between tag-matching criteria and specificity, modularity, robustness, and the process of duplication and divergence should be explored. 
Evolvability or information-theoretical analyses may prove fruitful in this regard \citep{tarapore2015evolvability}.
How to 
%apply insight into tag-matching criteria to 
systematically design new tag-matching metrics with desirable properties also remains an open problem.

Tag-like mechanisms play a central role mediating interaction and function across the spectrum of biological scale \citep{holland2012signals}.
% Our work suggests that tag-matching systems  can have implications with respect to the structure and evolution of interaction networks.
By shining light on previously-unexplored mechanistic and evolutionary properties of tagging systems, we hope that insight into artificial tag models will translate into a more nuanced appreciation of natural systems.
