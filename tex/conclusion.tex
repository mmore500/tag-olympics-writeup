\section{Conclusion}

Better understanding the mechanistic properties and functional implications of tag-matching criteria will help artificial life practitioners more effectively and incorporate tag-matching in model systems and better understand the biases imposed by those criteria.
In particular, our analyses suggests that tag-matching constraint, i.e., the degree of connectivity networks constructed through tag matching, is key to the interaction between a tag-matching scheme and a problem domain.
% Better understanding tag-matching criteria willOutside of artificial life, particularly in genetic programming, where increasing the rate of adaptive evolution and evolving better-quality solutions is a key concern.

However, open questions remain around tag-matching criteria.
In particular, the relationships between tag-matching criteria and specificity, modularity, robustness, and the process of duplication and divergence should be explored. 
Evolvability or information-theoretical analyses may prove fruitful in this regard \citep{tarapore2015evolvability}.
How to apply insight into tag-matching criteria to systematically design new metrics with desirable properties also remains an open question.

Tag-like mechanisms play a central role mediating interaction and function across the spectrum of biological scale \citep{holland2012signals}.
% Our work suggests that tag-matching systems  can have implications with respect to the structure and evolution of interaction networks.
By highlighting mechanistic and evolutionary properties of biological tags we might not have otherwise thought to investigate, we hope that insight into artificial tag models will translate into a more nuanced appreciation of natural systems.
